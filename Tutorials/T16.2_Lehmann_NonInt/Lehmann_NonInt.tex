\documentclass[11pt,a4paper,oneside]{article}

\usepackage{amsmath,amssymb,amsthm} 
\usepackage{graphicx}
\usepackage{enumerate}
%\usepackage{enumitem}
\usepackage{lastpage} % total page number
\usepackage[bottom]{footmisc} % to keep footnotes at the bottom of pages
\usepackage{braket}

\usepackage{fancyhdr}
\pagestyle{fancyplain}
%\pagestyle{fancy}
%\setlength{\headheight}{13.6pt}
%\setlength{\headsep}{7pt}
%\voffset-4.4mm
\renewcommand{\headrulewidth}{0.4pt}
\renewcommand{\footrulewidth}{0.4pt}

\voffset-4.4mm
\hoffset-4.4mm
\marginparsep0mm
\marginparwidth0mm
\oddsidemargin0mm
\evensidemargin0mm
\topmargin  0mm
\textwidth  168mm
\textheight 240mm %A4: 297mm x 210mm
\setlength{\footskip}{25pt}
\setlength{\headsep}{13pt}	%default 25pt
\setlength{\headheight}{12pt}	%def<ault 12pt ~4mm
\setlength{\headwidth}{\textwidth}

\lhead{{Tensor Network (2022)}}
\rhead{{\scriptsize Seung-Sup Lee}}
\lfoot{{\scriptsize }}
\cfoot{{\scriptsize Page \thepage\ of \pageref{LastPage}}}
\rfoot{{\scriptsize }}

\usepackage[sort,numbers,merge,sort&compress]{natbib} % mimick the option used in revtex
\newcommand*{\onlinecite}[1]{\citenum{#1}}

%\DeclareCiteCommand{\onlinecite}%[\mkbibbrackets] % cite without brackets
%  {\usebibmacro{cite:init}%
%   \usebibmacro{prenote}}
%  {\usebibmacro{citeindex}%
%   \usebibmacro{cite:comp}}
%  {}
%  {\usebibmacro{cite:dump}%
%   \usebibmacro{postnote}}

% reduce line spacing in the bibliography (source: https://www.math.cmu.edu/~gautam/sj/blog/20140712-bibtex-spacing.html )
\newlength{\bibitemsep}\setlength{\bibitemsep}{.2\baselineskip plus .05\baselineskip minus .05\baselineskip}
\newlength{\bibparskip}\setlength{\bibparskip}{0pt}
\let\oldthebibliography\thebibliography
\renewcommand\thebibliography[1]{%
  \oldthebibliography{#1}%
  \setlength{\parskip}{\bibitemsep}%
  \setlength{\itemsep}{\bibparskip}%
}

\usepackage[colorlinks=true,citecolor=blue,linkcolor=cyan,hypertexnames=false]{hyperref} % should be at the end of all the other packages

\begin{document}

\section*{Lehmann representation for a non-interacting system}

Consider a generic non-interacting system of fermions, whose Hamiltonian is quadratic, i.e.,
\begin{equation}
\hat{H} = \sum_{ij} [ \boldsymbol{h} ]_{ij} \hat{c}_{i}^\dagger \hat{c}_{j} ,
\end{equation}
where $\hat{c}_{i}^\dagger$ creates a fermionic particle in the $i$-th spin-orbital, and the spin-orbitals form the orthonormal basis.
The single-particle Hamiltonian $\boldsymbol{h}$, which is Hermitian, can be diagonalized as
\begin{equation}
\boldsymbol{h} = \boldsymbol{V} \boldsymbol{\epsilon} \boldsymbol{V}^\dagger,
\end{equation}
where $[\boldsymbol{\epsilon}]_{ij} = \epsilon_i \delta_{ij}$ is the diagonal matrix containing the single-particle energy eigenvalues $\epsilon_i$,
and $\boldsymbol{V} = ( \vec{v}_1 \vec{v}_2 \cdots )$ is the unitary matrix whose column vectors are the eigenvectors of $\boldsymbol{h}$.

Since the (many-body) Hamiltonian $\hat{H}$ is quadratic, the retarded Green's function,
\begin{equation}
G[\hat{c}_i, \hat{c}_j^\dagger] (t) = -i \theta (t) \, \mathrm{Tr}\!\left( \hat{\rho} \, [ \hat{c}_i (t), \hat{c}_j^\dagger ]_\pm ] \right),
\end{equation}
and its spectral function,
\begin{equation}
A[\hat{c}_i, \hat{c}_j^\dagger] (\omega) = \frac{-1}{\pi} \mathrm{Im} \int_{-\infty}^{+\infty} \mathrm{d}t \, e^{\mathrm{i} \omega t} \, G[\hat{c}_i, \hat{c}_j^\dagger] (t),
\end{equation}
can be computed exactly.

\begin{enumerate}[(a)]

\item
Evaluate the spectral function $A[\hat{c}_i, \hat{c}_j^\dagger] (\omega)$ by using the Lehmann representation.

\item
The local spectral function is the case of $i = j$ in which the two defining operators $\hat{c}_i$ and $\hat{c}_i^\dagger$ act on the same spin-orbital.
Explain why the local spectral function can be interpreted as the local density of states.

\end{enumerate}

\end{document}



