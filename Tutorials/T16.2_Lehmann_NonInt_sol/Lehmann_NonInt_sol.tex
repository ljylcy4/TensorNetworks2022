\documentclass[11pt,a4paper,oneside]{article}

\usepackage{amsmath,amssymb,amsthm} 
\usepackage{graphicx}
\usepackage{enumerate}
%\usepackage{enumitem}
\usepackage{lastpage} % total page number
\usepackage[bottom]{footmisc} % to keep footnotes at the bottom of pages
\usepackage{braket}

\usepackage{fancyhdr}
\pagestyle{fancyplain}
%\pagestyle{fancy}
%\setlength{\headheight}{13.6pt}
%\setlength{\headsep}{7pt}
%\voffset-4.4mm
\renewcommand{\headrulewidth}{0.4pt}
\renewcommand{\footrulewidth}{0.4pt}

\voffset-4.4mm
\hoffset-4.4mm
\marginparsep0mm
\marginparwidth0mm
\oddsidemargin0mm
\evensidemargin0mm
\topmargin  0mm
\textwidth  168mm
\textheight 240mm %A4: 297mm x 210mm
\setlength{\footskip}{25pt}
\setlength{\headsep}{13pt}	%default 25pt
\setlength{\headheight}{12pt}	%def<ault 12pt ~4mm
\setlength{\headwidth}{\textwidth}

\lhead{{Tensor Network (2022)}}
\rhead{{\scriptsize Seung-Sup Lee}}
\lfoot{{\scriptsize }}
\cfoot{{\scriptsize Page \thepage\ of \pageref{LastPage}}}
\rfoot{{\scriptsize }}

\usepackage[sort,numbers,merge,sort&compress]{natbib} % mimick the option used in revtex
\newcommand*{\onlinecite}[1]{\citenum{#1}}

%\DeclareCiteCommand{\onlinecite}%[\mkbibbrackets] % cite without brackets
%  {\usebibmacro{cite:init}%
%   \usebibmacro{prenote}}
%  {\usebibmacro{citeindex}%
%   \usebibmacro{cite:comp}}
%  {}
%  {\usebibmacro{cite:dump}%
%   \usebibmacro{postnote}}

% reduce line spacing in the bibliography (source: https://www.math.cmu.edu/~gautam/sj/blog/20140712-bibtex-spacing.html )
\newlength{\bibitemsep}\setlength{\bibitemsep}{.2\baselineskip plus .05\baselineskip minus .05\baselineskip}
\newlength{\bibparskip}\setlength{\bibparskip}{0pt}
\let\oldthebibliography\thebibliography
\renewcommand\thebibliography[1]{%
  \oldthebibliography{#1}%
  \setlength{\parskip}{\bibitemsep}%
  \setlength{\itemsep}{\bibparskip}%
}

\usepackage[colorlinks=true,citecolor=blue,linkcolor=cyan,hypertexnames=false]{hyperref} % should be at the end of all the other packages

\begin{document}

\section*{Lehmann representation for a non-interacting system}

Consider a generic non-interacting system of fermions, whose Hamiltonian is quadratic, i.e.,
\begin{equation}
\hat{H} = \sum_{ij} [ \boldsymbol{h} ]_{ij} \hat{c}_{i}^\dagger \hat{c}_{j} ,
\end{equation}
where $\hat{c}_{i}^\dagger$ creates a fermionic particle in the $i$-th spin-orbital, and the spin-orbitals form the orthonormal basis.
The single-particle Hamiltonian $\boldsymbol{h}$, which is Hermitian, can be diagonalized as
\begin{equation}
\boldsymbol{h} = \boldsymbol{V} \boldsymbol{\epsilon} \boldsymbol{V}^\dagger,
\label{eq:eig_h}
\end{equation}
where $[\boldsymbol{\epsilon}]_{ij} = \epsilon_i \delta_{ij}$ is the diagonal matrix containing the single-particle energy eigenvalues $\epsilon_i$,
and $\boldsymbol{V} = ( \vec{v}_1 \vec{v}_2 \cdots )$ is the unitary matrix whose column vectors are the eigenvectors of $\boldsymbol{h}$.

Since the (many-body) Hamiltonian $\hat{H}$ is quadratic, the retarded Green's function,
\begin{equation}
G[\hat{c}_i, \hat{c}_j^\dagger] (t) = -i \theta (t) \, \mathrm{Tr}\!\left( \hat{\rho} \, [ \hat{c}_i (t), \hat{c}_j^\dagger ]_\pm ] \right),
\end{equation}
and its spectral function,
\begin{equation}
A[\hat{c}_i, \hat{c}_j^\dagger] (\omega) = \frac{-1}{\pi} \mathrm{Im} \int_{-\infty}^{+\infty} \mathrm{d}t \, e^{\mathrm{i} \omega t} \, G[\hat{c}_i, \hat{c}_j^\dagger] (t),
\end{equation}
can be computed exactly.

\begin{enumerate}[(a)]

\item
Evaluate the spectral function $A[\hat{c}_i, \hat{c}_j^\dagger] (\omega)$ by using the Lehmann representation.

\textbf{[Solution]}
The diagonalized form of the Hamiltonian is 
\begin{equation}
\hat{H} = \sum_{k} \epsilon_k \hat{d}_k^\dagger \hat{d}_k = \sum_{k} \epsilon_k \hat{n}_{k},
\end{equation}
where $\hat{d}_k^\dagger = \sum_{i} \hat{c}_i^\dagger [\boldsymbol{V}]_{ik}$ creates a particle in the eigenmode of energy $\epsilon_k$ and $\hat{n}_{k} = \hat{d}_k^\dagger \hat{d}_k$ counts the particle number in the eigenmode.
(Here we call the single-particle energy eigenstates as eigenmodes, to distinguish from many-body energy eigenstates.)
The number operators $\hat{n}_{k}$ commute each other, and they also commute with the Hamiltonian.
Therefore, the eigenvalues of the number operators are good quantum numbers.
Indeed, we can represent the eigenstates as 
\begin{equation}
| \boldsymbol{n} \rangle = \cdots | n_{k+1} \rangle | n_{k} \rangle | n_{k-1} \rangle \cdots , 
\end{equation}
where $[\boldsymbol{n}]_k = n_k =  0,1$ is the eigenvalue of $\hat{n}_k$.
The energy eigenvalue corresponding to $|\boldsymbol{n} \rangle$ is $E_{\boldsymbol{n}} = \sum_k \epsilon_k n_k$.

Then, the Lehmann representation of $A[\hat{d}_k, \hat{d}_l^\dagger] (\omega)$ is given as follows:
\begin{equation}
\sum_{\boldsymbol{n}, \boldsymbol{n}'}
( \langle \boldsymbol{n} | \hat{\rho} | \boldsymbol{n} \rangle\!\langle\boldsymbol{n} | \hat{d}_k | \boldsymbol{n}' \rangle\!\langle \boldsymbol{n'} | \hat{d}_{l}^\dagger | \boldsymbol{n} \rangle
+
\langle \boldsymbol{n} | \hat{d}_k | \boldsymbol{n}' \rangle\!\langle \boldsymbol{n'} | \hat{\rho}  | \boldsymbol{n'} \rangle\!\langle \boldsymbol{n'} | \hat{d}_{l}^\dagger | \boldsymbol{n} \rangle )
\, \delta (\omega - (E_{\boldsymbol{n}'} - E_{\boldsymbol{n}})) .
\end{equation}
We see that $\langle\boldsymbol{n} | \hat{d}_k | \boldsymbol{n}' \rangle$ is finite only when $n'_k = 1$, $n_k = 0$, and $n'_m = n_m$ for $m \neq k$.
Similarly, $\langle \boldsymbol{n'} | \hat{d}_{l}^\dagger | \boldsymbol{n} \rangle$ is finite only when $n'_l = 1$, $n_l = 0$, and $n'_m = n_m$ for $m \neq l$.
Therefore, it should hold that $k = l$; otherwise $A[\hat{d}_k, \hat{d}_l^\dagger] (\omega) = 0$.

So we get
\begin{equation}
A[\hat{d}_k, \hat{d}_l^\dagger] (\omega) = 
\delta_{kl}
\sum_{\boldsymbol{n}, \boldsymbol{n}'}
(\langle \boldsymbol{n} | \hat{\rho} | \boldsymbol{n} \rangle + \langle \boldsymbol{n}' | \hat{\rho} | \boldsymbol{n}' \rangle)
\underbrace{| \langle\boldsymbol{n} | \hat{d}_k | \boldsymbol{n}' \rangle |^2}_{= 1}
\,
\, \delta (\omega - \underbrace{(E_{\boldsymbol{n}'} - E_{\boldsymbol{n}})}_{= \epsilon_k}) .
\end{equation}
The underbraced terms are independent from $\boldsymbol{n}$ and $\boldsymbol{n}'$ as long as they are finite.
So they can be pulled out of the summation $\sum_{\boldsymbol{n}, \boldsymbol{n}'}$,
\begin{equation}
A[\hat{d}_k, \hat{d}_l^\dagger] (\omega) = 
\delta_{kl}
\delta (\omega - \epsilon_k)
\underbrace{\sum_{\boldsymbol{n}, \boldsymbol{n}'}
(\langle \boldsymbol{n} | \hat{\rho} | \boldsymbol{n} \rangle + \langle \boldsymbol{n}' | \hat{\rho} | \boldsymbol{n}' \rangle)}_{= \sum_{\boldsymbol{n}''} \langle \boldsymbol{n}'' | \hat{\rho} | \boldsymbol{n}'' \rangle = 1} .
\end{equation}
The equality in this underbrace holds because the sets $\{ | \boldsymbol{n} \rangle \}$ and $\{ | \boldsymbol{n}' \rangle \}$ are exclusive, since $n_k \neq n'_k$, and their union equals to the set of all the eigenstates, since there is no further constraint other than $n_m = n'_m$ for $m \neq k$.

Therefore, we obtain the spectral function defined by $\hat{c}$ operators by applying the unitary transformation $\hat{c}_i^\dagger = \sum_{k} \hat{d}_k^\dagger [\boldsymbol{V}^{-1}]_{ki} =  \sum_{k} \hat{d}_k^\dagger [\boldsymbol{V}]_{ik}^*$,
which is clear from the structure of the Lehmann representation.
\begin{equation}
A[\hat{c}_i, \hat{c}_j^\dagger] (\omega) = \sum_k [\boldsymbol{V}]_{ik} [\boldsymbol{V}]_{jl}^* 
A[\hat{d}_k, \hat{d}_l^\dagger] (\omega)
= \sum_k [\boldsymbol{V}]_{ik} [\boldsymbol{V}]_{jk}^* 
\delta(\omega - \epsilon_k).
\end{equation}


\item
The local spectral function is the case of $i = j$ in which the two defining operators $\hat{c}_i$ and $\hat{c}_i^\dagger$ act on the same spin-orbital.
Explain why the local spectral function can be interpreted as the local density of states.

\textbf{[Solution]}
We have the local spectral function
\begin{equation}
A[\hat{c}_i, \hat{c}_i^\dagger] (\omega)
= \sum_k | [\boldsymbol{V}]_{ik} |^2
\delta(\omega - \epsilon_k).
\end{equation}
From the diagonalization of the single-particle Hamiltonian in Eq.~\eqref{eq:eig_h}, we can interpret $| [\boldsymbol{V}]_{ik} |^2$ as the probability that the particle at site $i$ is in the eigenmode $k$.
Of course, such probabilities are non-negative, $| [\boldsymbol{V}]_{ik} |^2 \geq 0$, and sum up to the unity, $\sum_k | [\boldsymbol{V}]_{ik} |^2 = 1$.
Therefore the local spectral function can be interpreted as the local density of states.

\end{enumerate}

\end{document}



